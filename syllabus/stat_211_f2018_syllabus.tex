\documentclass[11pt, a4paper]{article}
%\usepackage{geometry}
\usepackage[inner=1.5cm,outer=1.5cm,top=2.5cm,bottom=2.5cm]{geometry}
\pagestyle{empty}
\usepackage{graphicx}
\usepackage{fancyhdr, lastpage, bbding, pmboxdraw, booktabs}
\usepackage[usenames,dvipsnames]{color}
\definecolor{darkblue}{rgb}{0,0,.6}
\definecolor{darkred}{rgb}{.7,0,0}
\definecolor{darkgreen}{rgb}{0,.6,0}
\definecolor{red}{rgb}{.98,0,0}
\usepackage[colorlinks,pagebackref,pdfusetitle,urlcolor=darkblue,citecolor=darkblue,linkcolor=darkred,bookmarksnumbered,plainpages=false]{hyperref}
\renewcommand{\thefootnote}{\fnsymbol{footnote}}

\pagestyle{fancyplain}
\fancyhf{}
\lhead{ \fancyplain{}{Stat 211} }
%\chead{ \fancyplain{}{} }
\rhead{ \fancyplain{}{\today} }
%\rfoot{\fancyplain{}{page \thepage\ of \pageref{LastPage}}}
\fancyfoot[RO, LE] {page \thepage\ of \pageref{LastPage} }
\thispagestyle{plain}

%%%%%%%%%%%% LISTING %%%
\usepackage{listings}
\usepackage{caption}
\DeclareCaptionFont{white}{\color{white}}
\DeclareCaptionFormat{listing}{\colorbox{gray}{\parbox{\textwidth}{#1#2#3}}}
\captionsetup[lstlisting]{format=listing,labelfont=white,textfont=white}
\usepackage{verbatim} % used to display code
\usepackage{fancyvrb}
\usepackage{acronym}
\usepackage{amsmath}
\usepackage{amsthm}
\usepackage{bm}
\VerbatimFootnotes % Required, otherwise verbatim does not work in footnotes!



\definecolor{OliveGreen}{cmyk}{0.64,0,0.95,0.40}
\definecolor{CadetBlue}{cmyk}{0.62,0.57,0.23,0}
\definecolor{lightlightgray}{gray}{0.93}



\lstset{
%language=bash,                          % Code langugage
basicstyle=\ttfamily,                   % Code font, Examples: \footnotesize, \ttfamily
keywordstyle=\color{OliveGreen},        % Keywords font ('*' = uppercase)
commentstyle=\color{gray},              % Comments font
numbers=left,                           % Line nums position
numberstyle=\tiny,                      % Line-numbers fonts
stepnumber=1,                           % Step between two line-numbers
numbersep=5pt,                          % How far are line-numbers from code
backgroundcolor=\color{lightlightgray}, % Choose background color
frame=none,                             % A frame around the code
tabsize=2,                              % Default tab size
captionpos=t,                           % Caption-position = bottom
breaklines=true,                        % Automatic line breaking?
breakatwhitespace=false,                % Automatic breaks only at whitespace?
showspaces=false,                       % Dont make spaces visible
showtabs=false,                         % Dont make tabls visible
columns=flexible,                       % Column format
morekeywords={__global__, __device__},  % CUDA specific keywords
}

%%%%%%%%%%%%%%%%%%%%%%%%%%%%%%%%%%%%
\begin{document}

\begin{center}
{\Large \textsc{STAT 211 - 509: Principles of Statistics I}}
\end{center}
\begin{center}
Fall 2018
\end{center}

\begin{center}
\rule{\textwidth}{0.4pt}
\begin{minipage}[t]{\textwidth}
\begin{tabular}{llcccll}
\textbf{Instructor:} & Patrick Ding & & &  & \textbf{Email:} & \url{patrickding@stat.tamu.edu} 
\\
\textbf{Lectures:} &  Tue/Thu 12:45pm -- 2:00pm & & & & \textbf{Place:} & Blocker 150
\\
\textbf{Office hours:} & Mon/Wed 1:00pm -- 2:00pm   & & & & \textbf{Office:} & Blocker 455
\\\\
\textbf{TA:} & Junsouk Choi & & & & \textbf{Email:} & \url{jchoi@stat.tamu.edu}
\\
\textbf{Office hours:} & Wed 10:00am - 12:00pm   & & & & \textbf{Office:} & Blocker 420 
\end{tabular}
\end{minipage}
\rule{\textwidth}{0.4pt}
\end{center}
%\setlength{\unitlength}{1in}
%\renewcommand{\arraystretch}{2}

\begin{center}
\rule{\textwidth}{0.4pt}
\begin{minipage}[t]{\textwidth}
\begin{tabular}{llccll}
\textbf{Help Sessions:} & Grad students & & & \textbf{Place:} & Blocker 162
\\
\textbf{Times:} & Mon/Wed & & & & 10:15am-12:15pm, 1:45pm-3:45pm, 5:00pm-7:00pm
\\
 & Tue/Thu & & & & 10:15am-12:15pm, 2:00pm-4:00pm, 5:00pm-7:00pm
\end{tabular}
\end{minipage}
\rule{\textwidth}{0.4pt}
\end{center}

\section{Course Overview}

\noindent\textbf{Description:} 
Introduction to probability and probability distributions; sampling and descriptive measures; inference and hypothesis testing; linear regression, analysis of variance. 

\vskip.15in
\noindent\textbf{Prerequisites:}
MATH 152, 172 or instructor's permission. This course will use some calculus and is more math intensive than the
corresponding 3-0 levels of statistics. Knowledge of calculus is mandatory. To be
more specific, basic calculus is mandatory as of the first day of class and a working
knowledge of double integration will be a requirement by about 1/3 of the way through
the semester, for those taking MATH 172 concurrently. 

\vskip.15in
\noindent\textbf{Learning Outcomes:}
\begin{enumerate}
  \item Identify appropriate graphs, summary statistics, and inferential statistics for real-world contexts.
 \item Interpret graphs and statistics in real-world contexts.
 \item Calculate summary and inferential statistics.
 \item Infer appropriate conclusions about populations based on data.
 \item Explain and compare properties of summary and inferential statistics.
 \item Combine concepts in new ways to solve various problems.
\end{enumerate}

\vspace*{.15in}
\noindent \textbf{Course Outline:}

\begin{center}
\begin{tabular}{lllr}
\toprule
\bf Week & \bf Date & \bf Topic & \bf Homework 
\\
\midrule
1    & Aug 28, Aug 30  & Introduction and R tutorial & 
\\
2    & Sep 4, Sep 6 & Probability and random variables & HW 1 due Sep 7
\\
3    & Sep 11, Sep 13 & Introduction to statistical inference
\\
4    & Sep 18, Sep 20 & Expectation & HW 2 due Sep 17
\\
5    & Sep 25, Sep 27 & Conditional probability and Bayes' theorem & HW 3 due Sep 26
\\
6    & Oct 2, Oct 4 & Review, exam 1 during class & HW 4 due Oct 3
\\
7    & Oct 9, Oct 11 & Statistical inference with simulation
\\
8    & Oct 16, Oct 18 & Bayesian inference & HW 5 due Oct 15
\\
9    & Oct 23, Oct 25 & Continuous random variables & HW 6 due Oct 22
\\
10   & Oct 30, Nov 1 & Classical inference  & HW 7 due Oct 29
\\
11   & Nov 6, Nov 8 & Review, exam 2 during class
\\
12   & Nov 13, Nov 15 & Linear regression & HW 8 due Nov 14
\\
13   & Nov 20   & Analysis of variance & HW 9 due Nov 21
\\
14   & Nov 27, Nov 29 & Data production
\\
15   & Dec 4    & Review & HW 10 due Dec 3
\\
16   & Dec 12: 8:00am - 10:00am   & Final exam
\\
\bottomrule
\end{tabular}
\end{center}

% \begin{enumerate}
%     \item \texttt{Data collection and summarization}: populations and samples, frequency distributions, histograms, mean, median, variance, standard deviation, quartiles, interquartile range, boxplots
%     \item \texttt{Probability}: basic probability, conditional probability and independence, Bayes' theorem, basic reliability
%     \item \texttt{Discrete distributions}: mean and variance of a discrete distribution, binomial distribution, Poisson distribution and process 
%     \item \texttt{Continuous distributions}: normal distribution, gamma distribution
%     \item \texttt{Joint distributions and Central Limit Theorem}: joint distributions, Central Limit Theorem, sampling distribution of the sample mean
%     \item \texttt{Confidence intervals based on a single sample}: confidence interval for population mean, variance, proportion
%     \item \texttt{Hypothesis tests based on a single sample}: basics of hypothesis testing, hypothesis tests for population mean and population proportion
%     \item \texttt{Comparing two samples}: confidence intervals/hypothesis tests for two means, hypothesis test for two variances, confidence intervals/hypothesis tests for two proportions
%     \item \texttt{ANOVA}
%     \item \texttt{Linear Regression}: Least squares, hypothesis tests/confidence intervals/prediction intervals
% \end{enumerate}

\section{Course Resources}

\noindent\textbf{Main References:} %\footnotemark
There is no required textbook. You will be provided with lecture notes and other materials that will be
sufficient to support this course. An optional book to consider is \textit{Mathematical Statistics with Resampling and
R} by Chihara and Hesterberg; an electronic version of this book is available on the TAMU library website. A good beginner's resource for R is \textit{The
R Cookbook} by Paul Teetor; an electronic version of this book is available on the TAMU library website.


\vskip.15in
\noindent\textbf{Course Pages:} 
\begin{enumerate}
\item eCampus: Lecture notes, datasets, grades, and practice exams will be here.
\item Piazza: Rather than emailing questions to the teaching staff, I encourage you to post your questions on Piazza. Find our class page at: \url{https://piazza.com/tamu/fall2018/stat211509/home}.
\item Webassign: Homework will be here.
\end{enumerate}

\vskip.15in
\noindent\textbf{Software:} We will use the R programming language, available for download here: \href{https://www.r-project.org/}{https://www.r-project.org/}. I recommend you use Rstudio as your development environment: \href{https://www.rstudio.com/products/rstudio/download/}{https://www.rstudio.com}. 

\vskip.15in
\noindent\textbf{Help Sessions:} On Mondays through Thursdays at Blocker 162 you can get help on the course from STAT grad students. See the first page for times.

\section{Grading}

\noindent\textbf{Grading Policy:} 
Homework (30\%),  Midterm 1 (20\%), Midterm 2 (20\%), Final (30\%). 
\\
Grade cutoffs:
%
\begin{align*}
    100 \ge \bm{A} \ge 90 > \bm{B} \ge 80 > \bm{C} \ge 70 > \bm{D} \ge 60 > \bm{F} \ge 0
\end{align*}
%
For grades within 1\% of a higher grade I reserve the right to bump your grade up, provided you have consistently answered questions correctly on Piazza and/or have demonstrated improvement over time.

\vskip.15in
\noindent\textbf{Homework:}
Homework is posted and submitted via Webassign ( \url{https://www.webassign.net/
tamu/login.html} ). The cost of Webassign is \$22 and can be accessed and purchased by going to the eCampus site for this course and clicking on the link for Webassign on the left side.
All homeworks are due at 8:00 am. Check the course outline for due dates. \textbf{Late homework will never be accepted}, but the lowest homework score will be dropped. You are encouraged to work together, but the answers must be your own.

\vskip.15in
\noindent\textbf{Exams:}
If you know you will miss the exam for a valid reason, please notify me or the main office of the Department
of Statistics as soon as possible. For what constitutes a university excused absence, see \url{http://student-rules.tamu.edu/rule07}. All exams will be held in Blocker 150.
\begin{center} \begin{minipage}{3.8in}
\begin{flushleft}
Midterm \#1      \dotfill Oct 4  \\
Midterm \#2      \dotfill Nov 8  \\
%Project Deadline \dotfill ~Month Day \\
Final Exam       \dotfill Dec 12  \\
\end{flushleft}
\end{minipage}
\end{center}

\section{Class Policy}

\noindent\textbf{Attendance:}  
You are responsible for the material covered in lectures that you miss. If you miss a lecture, get the notes from someone who was in class. 


\vskip.15in
\noindent\textbf{Disability Accommodation:}  
The Americans with Disabilities Act (ADA) is a federal anti-discrimination statute that provides comprehensive civil rights protection for persons with disabilities. Among other things, this legislation requires that all students with disabilities be guaranteed a learning environment that provides for reasonable accommodation of their disabilities. If you believe you have a disability requiring an accommodation, please contact Disability Services, currently located in the Disability Services building at the Student Services at White Creek complex on west campus or call 979-845-1637. For additional information visit \url{http://disability.tamu.edu/}

\vskip.15in
\noindent\textbf{Plagiarism:} 
As commonly defined, plagiarism consists of passing off as one?s own ideas, words, writing, etc., which belong to another. In accordance with this definition, you are committing plagiarism if you copy the work of another person and turn it in as your own, even if you should have the permission of that person. Plagiarism is one of the worst academic sins, for the plagiarist destroys the trust among colleagues without which research cannot be safely communicated. If you have any questions regarding plagiarism, please consult the latest issue of the Texas A\&M University Student Rules, under the section ``Scholastic Dishonesty.''

\vskip.15in
\noindent\textbf{Academic Integrity:} ``An Aggie does not lie, cheat or steal, or tolerate those who do.'' Please refer to the Honor Council Rules and Procedures (\url{ http://aggiehonor.tamu.edu/ }) for more information on the honor code.

\vskip.15in
\noindent\textbf{Copyright Notice:}
The handouts used in this course are copyrighted. By ``handouts'', I mean all materials generated for this class, which include but are not limited to syllabi, quizzes, exams, lab problems, in-class materials, review sheets, and additional problem sets. Because these materials are copyrighted, you do not have the right to copy the handouts, unless I expressly grant permission.

%%%%%% THE END 
\end{document} 